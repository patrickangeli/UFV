\documentclass[a4paper,12pt]{article}
\usepackage[utf8]{inputenc}
\usepackage[portuguese]{babel}
\usepackage{amsmath}
\usepackage{graphicx}
\usepackage{hyperref}
\usepackage{geometry}
\geometry{margin=1in}
\usepackage{cite}
\usepackage{url}
\usepackage{ragged2e}
\renewcommand{\footnotesize}{\RaggedRight\small}

\title{Análise de Mercado para Startup de Intermediação de Carbono}
\author{}
\date{}

\begin{document}


\maketitle

\section{Estudo dos Clientes}

\subsection{Perfil do Comprador (Pequenas Empresas)}

\textbf{Segmentação e Características:}
\begin{itemize}
    \item O Sebrae-SP lançou uma \textit{Calculadora de Pegada de Carbono} específica para micro e pequenas empresas, indicando um mercado-alvo definido\footnote{\url{https://sp.agenciasebrae.com.br/dados/sebrae-sp-lanca-calculadora-de-emissao-de-carbono-para-pequenas-empresas/}}.
    \item \textbf{Motivações para Compensação de Carbono:}
    \begin{itemize}
        \item Reduzir gastos
        \item Atrair clientes que valorizam práticas ESG
        \item Conquistar novos mercados com selos ambientais
        \item Atender políticas institucionais ligadas ao processo de carbono\footnote{\url{https://sp.agenciasebrae.com.br/dados/sebrae-sp-lanca-calculadora-de-emissao-de-carbono-para-pequenas-empresas/}}
    \end{itemize}
\end{itemize}

\subsection{Perfil do Comprador (Grandes Mercados)}

\textbf{Motivações e Critérios:}
\begin{itemize}
    \item Atualmente, o mercado voluntário de carbono é o que gera maior retorno financeiro no Brasil.
    \item \textbf{Principais motivações:}
    \begin{itemize}
        \item Compensar emissões de gases de efeito estufa
        \item Cumprir metas internacionais de redução de emissões
        \item Potencial de movimentação financeira estimado em \textbf{US\$ 50 bilhões até 2030}\footnote{\url{https://valor.globo.com/conteudo-de-marca/b3/financas-sustentaveis/noticia/2024/03/06/brasil-pode-liderar-mercado-de-carbono-no-mundo-previsao-e-que-setor-movimente-us-50-bi-ate-2030.ghtml}}
    \end{itemize}
\end{itemize}

\section{Análise do Mercado}

\subsection{Tamanho e Potencial do Mercado}

\textbf{Projeções Econômicas:}
\begin{itemize}
    \item Estimativa de \textbf{US\$ 100 bilhões} em receitas de crédito de carbono até 2030.
    \item O Brasil concentra \textbf{15\% do potencial global de captura de carbono por meios naturais}\footnote{\url{https://valor.globo.com/conteudo-de-marca/b3/financas-sustentaveis/noticia/2024/03/06/brasil-pode-liderar-mercado-de-carbono-no-mundo-previsao-e-que-setor-movimente-us-50-bi-ate-2030.ghtml}}.
    \item Potencial de atender \textbf{48,7\% da demanda global de créditos de carbono}.
\end{itemize}

\subsection{Estrutura do Mercado}

\textbf{Regulamentação e Desenvolvimento:}
\begin{itemize}
    \item Mercado dividido em dois setores:
    \begin{enumerate}
        \item \textbf{Mercado Regulado}: Iniciativas do poder público.
        \item \textbf{Mercado Voluntário}: Iniciativa privada mais flexível.
    \end{enumerate}
    \item \textbf{Novidades Regulatórias:}
    \begin{itemize}
        \item Lei 15.042/2024 regulamenta o mercado de créditos de carbono no Brasil.
        \item Sistema Brasileiro de Comércio de Emissões de Gases de Efeito Estufa (SBCE) criado.
    \end{itemize}
\end{itemize}

\subsection{Tendências}

\textbf{Fatores de Crescimento:}
\begin{itemize}
    \item Crescente conscientização sobre sustentabilidade.
    \item Pressão por descarbonização da economia.
    \item Exigências de mercados internacionais.
    \item Potencial de atração de investimentos estrangeiros.
\end{itemize}

\section{Análise Competitiva}

\textbf{Intermediários Existentes:}
\begin{itemize}
    \item B3 (Bolsa de Valores) posicionando-se como hub para o mercado de carbono.
    \item Plataformas especializadas em registro e comercialização de créditos.
    \item Empresas como Biofílica Ambipar atuando no mercado.
\end{itemize}

\section{Estratégia de Entrada}

\textbf{Diferenciais Potenciais:}
\begin{itemize}
    \item Simplificação do processo de compensação para pequenas empresas.
    \item Tecnologia de fácil uso para cálculo e intermediação de créditos.
    \item Foco em transparência e rastreabilidade dos créditos.
\end{itemize}

\textbf{Considerações Importantes:}
\begin{itemize}
    \item Credibilidade será crucial para o sucesso no mercado\footnote{\url{https://valor.globo.com/brasil/noticia/2024/12/17/credibilidade-sera-crucial-para-o-mercado-de-carbono.ghtml}}.
    \item Necessário ir além da simples compra/venda de créditos, focando em \textbf{redução efetiva de emissões}.
\end{itemize}

\section{Conclusão}

O mercado brasileiro de créditos de carbono está em \textbf{fase de expansão}, com regulamentação recente e \textbf{potencial significativo de crescimento}. Uma startup de intermediação tem \textbf{oportunidades promissoras}, especialmente ao focar em:
\begin{itemize}
    \item Simplificação do processo para pequenas empresas
    \item Transparência na intermediação
    \item Tecnologia de fácil uso
    \item Suporte na redução efetiva de emissões
\end{itemize}



\end{document}
