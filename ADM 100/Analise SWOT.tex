\documentclass[12pt,a4paper]{article}
\usepackage[utf8]{inputenc}
\usepackage[T1]{fontenc}
\usepackage[brazilian]{babel}
\usepackage{array}

\begin{document}
\title{Análise SWOT - Empreendimento de Redução de Emissões de Carbono com IA}
\author{Análise Estratégica}
\date{\today}
\maketitle
\begin{table}[h]
    \centering
    \begin{tabular}{|p{0.45\textwidth}|p{0.45\textwidth}|}
    \hline
    \textbf{Forças (Strengths)} & \textbf{Fraquezas (Weaknesses)} \\
    \hline
    • Alto investimento inicial em tecnologia e infraestrutura \\
    • Necessidade de equipe altamente especializada \\
    • Dependência de certificações e validações externas \\
    • Complexidade na implementação do sistema
    &
    • Uso de tecnologia avançada (IA) para monitoramento preciso de emissões \\
    • Automatização de processos de medição e verificação \\
    • Capacidade de fornecer dados em tempo real \\
    • Conformidade com o Sistema Brasileiro de Comércio de Emissões \\
    \hline
    \textbf{Oportunidades (Opportunities)} & \textbf{Ameaças (Threats)} \\
    \hline
    • Mercado de carbono em expansão no Brasil \\
    • Crescente demanda por soluções sustentáveis \\
    • Nova regulamentação do mercado de carbono brasileiro \\
    • Acesso a financiamentos verdes \\
    • Potencial de expansão internacional
    &
    • Volatilidade do mercado de carbono \\
    • Mudanças regulatórias frequentes \\
    • Concorrência de empresas tradicionais do setor \\
    • Resistência de empresas à adoção de novas tecnologias \\
    • Instabilidade econômica global \\
    \hline
    \end{tabular}
    \caption{Análise SWOT - Empreendimento de Redução de Emissões de Carbono com IA}
    \label{tab:swot}
    \end{table}
\vspace{1cm}
\textbf{Observações:}
\begin{itemize}
\item Esta análise SWOT foi desenvolvida considerando o cenário atual do mercado de carbono no Brasil
\item Os fatores listados podem variar dependendo do contexto regional e temporal
\item Recomenda-se revisão periódica desta análise para manter sua relevância
\end{itemize}
\end{document}