\documentclass{article}
\usepackage[utf8]{inputenc}
\usepackage{enumitem}
\usepackage{amsmath}
\usepackage{titlesec}

\begin{document}

\title{Resumo P2}
\author{Patrick de Angeli}
\maketitle

\tableofcontents

\newpage

\section{Plano de Negócios}

\subsection{Definição}
Um plano de negócios é um documento crucial para qualquer empreendimento, servindo como um guia detalhado que descreve os objetivos do negócio e os passos necessários para alcançá-los. Ele auxilia na identificação e mitigação de riscos e incertezas, aumentando as chances de sucesso do negócio.

\subsection{Objetivos}
\begin{itemize}
    \item Testar a viabilidade da ideia, verificando se ela possui potencial de retorno econômico.
    \item Orientar o desenvolvimento das operações e da estratégia, definindo o caminho a ser seguido pela empresa.
    \item Atrair recursos financeiros, demonstrando a solidez do negócio para potenciais investidores.
    \item Transmitir credibilidade para stakeholders, como bancos, investidores e parceiros.
    \item Desenvolver a equipe de gestão, alinhando a visão e os objetivos do negócio a todos os membros da equipe.
\end{itemize}

\subsection{Estrutura}
\begin{itemize}
    \item \textbf{Sumário Executivo:} Apresentação concisa dos principais pontos do plano.
    \item \textbf{Descrição da Empresa:} Detalhes sobre a empresa, como missão, visão, valores, estrutura organizacional, histórico e localização.
    \item \textbf{Produtos e Serviços:} Descrição detalhada dos produtos e serviços oferecidos, incluindo seus diferenciais e vantagens competitivas.
    \item \textbf{Análise de Mercado:} Análise do mercado em que a empresa atua, incluindo a identificação do público-alvo, a análise da concorrência, as tendências de mercado e as oportunidades e ameaças.
    \item \textbf{Plano de Marketing:} Estratégias para alcançar o mercado-alvo, incluindo as estratégias de produto, preço, praça (distribuição) e promoção.
    \item \textbf{Plano Operacional:} Descrição detalhada de como a empresa irá operar, incluindo os processos de produção, a gestão de estoques, a logística e a infraestrutura.
    \item \textbf{Plano Financeiro:} Projeções financeiras da empresa, incluindo as receitas, os custos, os investimentos, o fluxo de caixa e os indicadores de rentabilidade.
    \item \textbf{Plano de Recursos Humanos:} Estratégias para a gestão de pessoas, incluindo o recrutamento, a seleção, o treinamento, o desenvolvimento, a remuneração e os benefícios.
    \item \textbf{Análise de Riscos:} Identificação e análise dos principais riscos que a empresa enfrenta, bem como as estratégias para mitigá-los.
    \item \textbf{Plano de Sustentabilidade:} Estratégias que visam garantir que a empresa tenha um impacto ambiental positivo ou neutro, incluindo o uso eficiente de recursos naturais, redução de emissões de carbono e responsabilidade social.
\end{itemize}

\section{Marketing}

\subsection{Definição}
\textbf{Marketing} é o processo de criação, comunicação e entrega de valor para os clientes e para a gestão de relacionamentos com eles de forma que beneficiem a organização e seus stakeholders.

\subsection{Processo de Marketing}
\begin{itemize}
    \item \textbf{Compreender o mercado e as necessidades dos clientes:} Através de pesquisa de mercado, o profissional de marketing precisa identificar as necessidades e desejos dos seus clientes, bem como as tendências e oportunidades do mercado.
    \item \textbf{Desenvolver uma estratégia de marketing orientada para o cliente:} Com base nas informações coletadas na etapa anterior, o profissional de marketing deve desenvolver uma estratégia de marketing que atenda às necessidades dos clientes e os posicione em relação à concorrência.
    \item \textbf{Construir um programa de marketing integrado que entregue valor superior:} O programa de marketing deve incluir um mix de marketing eficaz, que combine produto, preço, praça e promoção de forma sinérgica.
    \item \textbf{Construir relacionamentos lucrativos com os clientes:} O objetivo do marketing é construir relacionamentos duradouros com os clientes, que gerem valor para ambas as partes.
    \item \textbf{Capturar valor dos clientes para criar lucros e valor para o cliente:} O sucesso do marketing é medido pela capacidade da empresa de capturar valor dos clientes, gerando lucro e satisfação.
\end{itemize}

\section{Operações}

\subsection{Definição}
\textbf{Administração de operações} é a área de administração que trata da gestão dos recursos e das atividades necessárias para a produção de bens e serviços.

\subsection{Processo de Transformação}

O processo de transformação é o núcleo da \textbf{Administração de Operações}, responsável por converter \textbf{insumos} em \textbf{produtos} ou \textbf{serviços}. Esse processo permeia toda a organização, impactando e sendo impactado por outras áreas funcionais. A visão sistêmica é crucial para a compreensão da administração de operações e do processo de transformação.

O processo de transformação pode ser compreendido em três etapas principais:

\textbf{1. Entrada (Input):}

Compreende os recursos que serão utilizados no processo de transformação. Eles podem ser classificados em duas categorias:

\begin{itemize}
\item \textbf{Recursos Transformados:} São os recursos que sofrem alterações durante o processo.
    \begin{itemize}
    \item Exemplos: Matérias-primas em uma fábrica, pacientes em um hospital, informações em um sistema contábil.
    \end{itemize}
\item \textbf{Recursos de Transformação:} São os recursos que atuam sobre os recursos transformados.
    \begin{itemize}
    \item Exemplos: Instalações, equipamentos, tecnologias, trabalhadores.
    \end{itemize}
\end{itemize}

\textbf{2. Processo de Transformação:}

É a etapa onde os insumos são convertidos em produtos ou serviços. Os tipos de processamento variam de acordo com a natureza do recurso transformado.

\begin{itemize}
\item \textbf{Processamento de Materiais:} Envolve a alteração das propriedades físicas dos materiais.
    \begin{itemize}
    \item Exemplos: Indústrias manufatureiras, empresas de transporte, empresas de armazenagem.
    \end{itemize}
\item \textbf{Processamento de Informações:} Envolve a modificação das características, posse, estocagem ou localização das informações.
    \begin{itemize}
    \item Exemplos: Empresas de contabilidade, institutos de pesquisa de mercado, bibliotecas, empresas de telecomunicações.
    \end{itemize}
\item \textbf{Processamento de Consumidores:} Envolve a alteração da localização, estado físico ou psicológico, ou acomodação dos consumidores.
    \begin{itemize}
    \item Exemplos: Empresas de turismo, hospitais, hotéis.
    \end{itemize}
\end{itemize}

\textbf{3. Saída (Output):}

Consiste nos bens ou serviços resultantes do processo de transformação. As características dos outputs variam de acordo com o tipo de processo e os insumos utilizados.

\begin{itemize}
\item \textbf{Tangibilidade:} Grau em que o output pode ser tocado.
\item \textbf{Estocabilidade:} Capacidade de armazenar o output.
\item \textbf{Transportabilidade:} Facilidade de transportar o output.
\item \textbf{Qualidade:} Grau de conformidade com as especificações e expectativas do cliente.
\end{itemize}

\section{Finanças}

\subsection{Definição}
\textbf{Administração financeira} é a área da organização responsável por gerir os recursos financeiros, garantindo sua utilização eficiente para atingir os objetivos organizacionais. Ela abrange decisões relacionadas a investimentos, financiamento e gestão de ativos, sempre buscando o equilíbrio entre risco e retorno.

\subsection{Funções Principais}
\begin{itemize}
\item \textbf{Planejamento Financeiro:} Definir metas financeiras e estratégias para alcançá-las.
\item \textbf{Gestão de Capital de Giro:} Garantir que a organização tenha recursos suficientes para suas operações diárias.
\item \textbf{Análise de Investimentos:} Avaliar projetos e oportunidades para determinar sua viabilidade e retorno esperado.
\item \textbf{Captação de Recursos:} Buscar fontes de financiamento adequadas, como empréstimos, emissão de ações ou reinvestimento de lucros.
\item \textbf{Controle Financeiro:} Monitorar receitas, despesas e desempenho financeiro para garantir a saúde financeira da organização.
\end{itemize}

\end{document}