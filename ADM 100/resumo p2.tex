\documentclass{article}
\usepackage{amsmath}
\usepackage{amssymb}
\usepackage[utf8]{inputenc}



\title{Resumo P2}
\author{Patrick de Angeli}
\date{}

\begin{document}

\maketitle
\tableofcontents  % Adiciona o sumário aqui
\newpage          % Opcional: cria uma nova página após o sumário


\section{Plano de Negócios}

\subsection{Definição}
Um plano de negócios é um documento crucial para qualquer empreendimento, servindo como um guia detalhado que descreve os objetivos do negócio e os passos necessários para alcançá-los. Ele auxilia na identificação e mitigação de riscos e incertezas, aumentando as chances de sucesso do negócio.

\subsection{Objetivos}
\begin{itemize}
    \item Testar a viabilidade da ideia, verificando se ela possui potencial de retorno econômico
    \item Orientar o desenvolvimento das operações e da estratégia, definindo o caminho a ser seguido pela empresa
    \item Atrair recursos financeiros, demonstrando a solidez do negócio para potenciais investidores
    \item Transmitir credibilidade para stakeholders, como bancos, investidores e parceiros
    \item Desenvolver a equipe de gestão, alinhando a visão e os objetivos do negócio a todos os membros da equipe
\end{itemize}

\subsection{Estrutura}
\begin{itemize}
    \item \textbf{Sumário Executivo:} Apresentação concisa dos principais pontos do plano
    \item \textbf{Conceito do Negócio:} Descrição detalhada da ideia do negócio
    \item \textbf{Análise de Mercado:} Estudo do mercado em que a empresa irá atuar
\end{itemize}

\section{Marketing}

\subsection{Definição}
Marketing é um processo que envolve a criação, comunicação e entrega de valor para os clientes, com o objetivo de construir relacionamentos lucrativos e duradouros.

\subsection{Processo de Marketing}
\begin{itemize}
    \item Entender o mercado e as necessidades dos clientes
    \item Elaborar uma estratégia de marketing orientada para o cliente
    \item Desenvolver um programa de marketing integrado
    \item Construir relacionamentos lucrativos
    \item Capturar valor dos clientes
\end{itemize}

\section{Operações}

\subsection{Definição}
Gestão de Operações é a área responsável por planejar, organizar e controlar os processos de transformação de insumos em produtos e serviços.

\subsection{Processo de Transformação}
\begin{itemize}
    \item \textbf{Inputs:} Recursos utilizados no processo produtivo
    \item \textbf{Processo de Transformação:} Conversão dos inputs em outputs
    \item \textbf{Outputs:} Resultados do processo produtivo
\end{itemize}

\section{Finanças}

\subsection{Definição}
A área financeira é responsável pela gestão dos recursos financeiros da empresa, incluindo planejamento, captação, aplicação e controle do dinheiro.

\subsection{Classificação dos Gastos}
\begin{itemize}
    \item \textbf{Investimentos:} Gastos ativados no balanço patrimonial
    \item \textbf{Custos:} Gastos relacionados diretamente com a produção
    \item \textbf{Despesas:} Gastos relacionados com administração e gestão
\end{itemize}

\end{document}
