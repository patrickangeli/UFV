\documentclass{article}
\usepackage[utf8]{inputenc}
\usepackage{enumitem}
\usepackage{amsmath}
\usepackage{titlesec}

\begin{document}

\title{Resumo P2}
\author{Patrick de Angeli}
\maketitle

\tableofcontents

\section{Plano de Negócios}

\subsection{Definição}
Um plano de negócios é um documento crucial para qualquer empreendimento, servindo como um guia detalhado que descreve os objetivos do negócio e os passos necessários para alcançá-los. Ele auxilia na identificação e mitigação de riscos e incertezas, aumentando as chances de sucesso do negócio.

\subsection{Objetivos}
\begin{itemize}
    \item Testar a viabilidade da ideia, verificando se ela possui potencial de retorno econômico.
    \item Orientar o desenvolvimento das operações e da estratégia, definindo o caminho a ser seguido pela empresa.
    \item Atrair recursos financeiros, demonstrando a solidez do negócio para potenciais investidores.
    \item Transmitir credibilidade para stakeholders, como bancos, investidores e parceiros.
    \item Desenvolver a equipe de gestão, alinhando a visão e os objetivos do negócio a todos os membros da equipe.
\end{itemize}

\subsection{Estrutura}
\begin{itemize}
    \item \textbf{Sumário Executivo:} Apresentação concisa dos principais pontos do plano.
    \item \textbf{Descrição da Empresa:} Detalhes sobre a empresa, como missão, visão, valores, estrutura organizacional, histórico e localização.
    \item \textbf{Produtos e Serviços:} Descrição detalhada dos produtos e serviços oferecidos, incluindo seus diferenciais e vantagens competitivas.
    \item \textbf{Análise de Mercado:} Análise do mercado em que a empresa atua, incluindo a identificação do público-alvo, a análise da concorrência, as tendências de mercado e as oportunidades e ameaças.
    \item \textbf{Plano de Marketing:} Estratégias para alcançar o mercado-alvo, incluindo as estratégias de produto, preço, praça (distribuição) e promoção.
    \item \textbf{Plano Operacional:} Descrição detalhada de como a empresa irá operar, incluindo os processos de produção, a gestão de estoques, a logística e a infraestrutura.
    \item \textbf{Plano Financeiro:} Projeções financeiras da empresa, incluindo as receitas, os custos, os investimentos, o fluxo de caixa e os indicadores de rentabilidade.
    \item \textbf{Plano de Recursos Humanos:} Estratégias para a gestão de pessoas, incluindo o recrutamento, a seleção, o treinamento, o desenvolvimento, a remuneração e os benefícios.
    \item \textbf{Análise de Riscos:} Identificação e análise dos principais riscos que a empresa enfrenta, bem como as estratégias para mitigá-los.
\end{itemize}

\section{Marketing}

\subsection{Definição}
\textbf{Marketing} é o processo de criação, comunicação e entrega de valor para os clientes e para a gestão de relacionamentos com eles de forma que beneficiem a organização e seus stakeholders.

\subsection{Processo de Marketing}
\begin{itemize}
    \item \textbf{Compreender o mercado e as necessidades dos clientes:} Através de pesquisa de mercado, o profissional de marketing precisa identificar as necessidades e desejos dos seus clientes, bem como as tendências e oportunidades do mercado.
    \item \textbf{Desenvolver uma estratégia de marketing orientada para o cliente:} Com base nas informações coletadas na etapa anterior, o profissional de marketing deve desenvolver uma estratégia de marketing que atenda às necessidades dos clientes e os posicione em relação à concorrência.
    \item \textbf{Construir um programa de marketing integrado que entregue valor superior:} O programa de marketing deve incluir um mix de marketing eficaz, que combine produto, preço, praça e promoção de forma sinérgica.
    \item \textbf{Construir relacionamentos lucrativos com os clientes:} O objetivo do marketing é construir relacionamentos duradouros com os clientes, que gerem valor para ambas as partes.
    \item \textbf{Capturar valor dos clientes para criar lucros e valor para o cliente:} O sucesso do marketing é medido pela capacidade da empresa de capturar valor dos clientes, gerando lucro e satisfação.
\end{itemize}

\section{Operações}

\subsection{Definição}
\textbf{Administração de operações} é a área de administração que trata da gestão dos recursos e das atividades necessárias para a produção de bens e serviços.

\subsection{Processo de Transformação}
O processo de transformação envolve a conversão de insumos (matérias-primas, energia, mão de obra) em produtos acabados. As principais etapas do processo de transformação são:
\begin{itemize}
    \item \textbf{Planejamento:} Definição dos objetivos de produção, dos recursos necessários e do cronograma de produção.
    \item \textbf{Organização:} Alocação dos recursos, definição dos fluxos de trabalho e estabelecimento dos procedimentos operacionais.
    \item \textbf{Direção:} Liderança da equipe de produção, acompanhamento do processo produtivo e tomada de medidas corretivas.
    \item \textbf{Controle:} Monitoramento do desempenho da produção, comparação com os objetivos estabelecidos e implementação de ações de melhoria.
\end{itemize}

\section{Finanças}

\subsection{Definição}
\textbf{Administração financeira} é a área da organização responsável pela gestão do fluxo de recursos financeiros.

\subsection{Classificação dos Gastos}
Os gastos de uma empresa podem ser classificados em:
\begin{itemize}
    \item \textbf{Custos fixos:} Gastos que não variam com o volume de produção, como aluguel e salários da administração.
    \item \textbf{Custos variáveis:} Gastos que variam com o volume de produção, como matérias-primas e energia.
    \item \textbf{Despesas operacionais:} Gastos relacionados com a administração da empresa, como marketing e vendas.
    \item \textbf{Despesas financeiras:} Gastos relacionados com a captação de recursos, como juros e comissões.
\end{itemize}

\subsection{Demonstrativos Financeiros}
As demonstrações financeiras são relatórios que apresentam a situação econômica e financeira da empresa. As principais demonstrações financeiras são:
\begin{itemize}
    \item \textbf{Balanço patrimonial:} Apresenta os ativos, os passivos e o patrimônio líquido da empresa em um determinado momento.
    \item \textbf{Demonstração do resultado do exercício (DRE):} Apresenta as receitas, os custos e os lucros da empresa em um determinado período.
    \item \textbf{Demonstração do fluxo de caixa (DFC):} Apresenta as entradas e saídas de caixa da empresa em um determinado período.
\end{itemize}

\section{Recursos Humanos}

\subsection{Definição}
\textbf{Administração de recursos humanos (ARH)} é a área da administração que trata da gestão das pessoas nas organizações.

\subsection{Processos da ARH}
Os principais processos da ARH são:
\begin{itemize}
    \item \textbf{Planejamento de recursos humanos:} Definição das necessidades de mão de obra da empresa.
    \item \textbf{Recrutamento e seleção:} Atração e escolha de candidatos para as vagas disponíveis.
    \item \textbf{Treinamento e desenvolvimento:} Capacitação dos funcionários para o desempenho de suas funções.
    \item \textbf{Avaliação de desempenho:} Medição do desempenho dos funcionários.
    \item \textbf{Remuneração e benefícios:} Definição dos salários e benefícios dos funcionários.
    \item \textbf{Relações trabalhistas:} Gestão do relacionamento entre a empresa e os sindicatos.
\end{itemize}

\subsection{Importância da ARH}
A ARH é fundamental para o sucesso das organizações, pois as pessoas são o seu principal ativo. Uma gestão eficaz de recursos humanos pode contribuir para:
\begin{itemize}
    \item \textbf{Aumento da produtividade:} Funcionários motivados e capacitados produzem mais e melhor.
    \item \textbf{Melhoria da qualidade dos produtos e serviços:} Funcionários qualificados e engajados produzem produtos e serviços de melhor qualidade.
    \item \textbf{Redução dos custos:} Uma gestão de recursos humanos eficiente pode reduzir os custos com rotatividade, absenteísmo e acidentes de trabalho.
    \item \textbf{Melhoria do clima organizacional:} Um ambiente de trabalho positivo e motivador contribui para a satisfação dos funcionários e para a retenção de talentos.
\end{itemize}

\end{document}