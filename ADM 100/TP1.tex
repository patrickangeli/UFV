\documentclass[11pt]{article}
\usepackage[utf8]{inputenc}
\usepackage[T1]{fontenc}
\usepackage[brazilian]{babel}  % Use 'brazilian' em vez de 'portuguese'
\usepackage{mathptmx}         % Para fonte Times New Roman
\usepackage[left=3cm,right=3cm,top=2.5cm,bottom=2.5cm]{geometry}
\usepackage{setspace}
\usepackage{tikz}
\usetikzlibrary{shapes,arrows,positioning}


\begin{document}

\begin{center}
    \textbf{ADM100 - Processo Decisório em Administração} \\
    Prof. Bruno Tavares \\
    Aluno: Patrick de Angeli Almeida - 98901 \\
    Pedro Henrique Lopes de Almeida - 108203 \\
    Rafael Martins Caetite Lopes Cançado - 108192  \\
    \vspace{1cm}
    \textbf{Pergunta Norteadora:} Como aplicar as ferramentas de processo decisório para solucionar os problemas de comunicação com clientes na empresa de Sônia?
\end{center}

\section{Tipos de Decisões no Caso de Sônia}

\subsection{Decisões Programadas vs. Não Programadas}

Sônia enfrenta uma mistura de decisões programadas e não programadas:

\begin{itemize}
    \item \textbf{Decisões Programadas}: Rotinas diárias de atendimento ao cliente via WhatsApp, gerenciamento de pedidos.
    \item \textbf{Decisões Não Programadas}: Lidar com o rápido crescimento, escassez de mão de obra qualificada, estruturação administrativa.
\end{itemize}

\section{Processo Decisório Aplicado ao Caso}

\begin{enumerate}
    \item \textbf{Identificação do problema}: Crescimento acelerado gerando desafios de gestão.
    \item \textbf{Diagnóstico}:
    \begin{itemize}
        \item Escassez de mão de obra qualificada
        \item Problemas de comunicação com clientes
        \item Falta de estrutura administrativa
    \end{itemize}
    \item \textbf{Desenvolvimento de alternativas}:
    \begin{itemize}
        \item Melhorar processos de recrutamento e seleção
        \item Diversificar canais de comunicação
        \item Implementar sistema de gestão (ERP)
    \end{itemize}
\end{enumerate}

\section{Técnicas de Diagnóstico}

\subsection{Diagrama de Ishikawa}



%\begin{document}
\begin{tikzpicture}[
    node distance=2cm,
    problema/.style={rectangle, draw, fill=white, text width=3cm, text centered, minimum height=1cm},
    categoria/.style={rectangle, draw, fill=white, text width=2cm, text centered, minimum height=0.75cm},
    causa/.style={rectangle, draw, fill=white, text width=3cm, text centered, minimum height=0.75cm}
]

% Problema central
\node[problema] (problema) {Problemas de comunicação com clientes};

% Categorias principais
\node[categoria] (material) at (5,3) {Material};
\node[categoria] (metodo) at (5,1) {Método};
\node[categoria] (maquina) at (5,-1) {Máquina};
\node[categoria] (mao) at (5,-3) {Mão-de-obra};

% Causas específicas
\node[causa] (causa1) at (9,3) {Falta de scripts ou FAQs para respostas rápidas};
\node[causa] (causa2) at (9,1) {Processos de atendimento ineficientes};
\node[causa] (causa3) at (9,-1) {Limitações do WhatsApp como único canal};
\node[causa] (causa4) at (9,-3) {Falta de treinamento em atendimento};

% Conexões
\draw[->] (problema) -- (material);
\draw[->] (problema) -- (metodo);
\draw[->] (problema) -- (maquina);
\draw[->] (problema) -- (mao);

\draw[->] (material) -- (causa1);
\draw[->] (metodo) -- (causa2);
\draw[->] (maquina) -- (causa3);
\draw[->] (mao) -- (causa4);

\end{tikzpicture}
%\end{document}


\subsection{Princípio de Pareto}
Aplicar este princípio para identificar os 20\% das causas que geram 80\% dos problemas na empresa, priorizando ações para solucionar as questões mais críticas.

\begin{enumerate}
    \item \textbf{Demora nas entregas e custos altos} (40\%)
    \begin{itemize}
        \item Múltiplas viagens dos entregadores;
        \item Dificuldade em encontrar entregadores disponíveis;
        \item Aumento significativo dos custos logísticos.
    \end{itemize}
    \item \textbf{Comunicação inadequada dos consultores do sistema} (25\%)
    \begin{itemize}
        \item Uso excessivo de gírias de internet;
        \item Substituição de palavras por expressões informais;
        \item Tempo limitado para responder a todos os clientes.
    \end{itemize}
    \item \textbf{Problemas de infraestrutura humana} (20\%)
    \begin{itemize}
        \item Sobrecarga dos funcionários;
        \item Comunicação ineficiente entre equipe;
        \item Contratações baseadas em indicações internas.
    \end{itemize}
\end{enumerate}



Aplicando o princípio de Pareto, esses três problemas representam 85\% dos desafios da empresa, concentrando-se principalmente em:
\begin{itemize}
    \item Logística de entrega;
    \item Comunicação do sistema;
    \item Gestão de recursos humanos.
\end{itemize}

\subsection*{Recomendações Estratégicas}

\begin{enumerate}
    \item \textbf{Otimização Logística}
    \begin{itemize}
        \item Implementar sistema de agrupamento de entregas;
        \item Criar rotas mais eficientes;
        \item Negociar tarifas fixas com entregadores.
    \end{itemize}
    \item \textbf{Melhoria na Comunicação}
    \begin{itemize}
        \item Padronizar linguagem de comunicação;
        \item Criar manual de comunicação;
        \item Implementar treinamento de comunicação profissional.
    \end{itemize}
    \item \textbf{Infraestrutura Humana}
    \begin{itemize}
        \item Desenvolver processos de comunicação interna;
        \item Criar sistema de feedback contínuo;
        \item Investir em treinamento e desenvolvimento.
    \end{itemize}
\end{enumerate}

\section{Desenvolvimento e Avaliação de Alternativas}

Sônia deve desenvolver alternativas abrangentes, genuínas, exequíveis e numerosas para cada desafio. Por exemplo, para o problema de comunicação, ela poderia considerar:

\begin{enumerate}
    \item Implementar um chatbot
    \item Criar um sistema de FAQ no site
    \item Contratar mais funcionários para atendimento
    \item Terceirizar o serviço de atendimento ao cliente
\end{enumerate}

\section{Teoria da Racionalidade Limitada}

É importante que Sônia reconheça as limitações da racionalidade na tomada de decisões. Ela deve estar ciente de que nem sempre terá todas as informações necessárias e que fatores como tempo e recursos limitados influenciarão suas escolhas.

\section{Conclusão}

Ao aplicar esses conceitos e ferramentas de tomada de decisão, Sônia poderá abordar os desafios de sua empresa de comércio virtual de forma mais estruturada e eficaz, aumentando suas chances de sucesso no gerenciamento do crescimento acelerado.

\end{document}