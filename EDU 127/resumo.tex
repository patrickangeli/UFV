\section{Resumo}


\subsection{Indutivismo}

\begin{itemize}
    \item Limitações do indutivismo ingênuo:
    \begin{itemize}
        \item Problema da indução: a justificação da indução é circular, uma vez que utiliza o próprio método indutivo para validá-lo. A experiência passada não garante a validade futura.
        \item Vagueza na aplicação: o princípio da indução é vago em relação à quantidade de observações necessárias para uma generalização válida.
        \item Dependência da teoria: as observações que sustentam o indutivismo não são neutras, pois as proposições de observação são formuladas em linguagem teórica.
    \end{itemize}
    \item Indutivismo sofisticado:
    \begin{itemize}
        \item Reconhece que a ciência não começa com a observação pura, mas envolve conjecturas criativas e a justificação através da corroboração indutiva.
        \item Ainda é relevante por levantar questões importantes sobre a dependência da teoria em relação à observação e a falibilidade das proposições de observação.
    \end{itemize}
    \item Continuidade das discussões:
    \begin{itemize}
        \item O indutivismo não oferece uma explicação satisfatória da ciência, sendo um ponto de partida para abordagens mais complexas, como os paradigmas de Kuhn e os programas de pesquisa de Lakatos.
    \end{itemize}
\end{itemize}

\subsection{Falsificacionismo}

\begin{itemize}
    \item Princípios fundamentais:
    \begin{itemize}
        \item Falseabilidade: para ser científica, uma teoria deve ser falsificável, ou seja, deve ser possível concebê-la sendo refutada por observações ou experimentos.
        \item Ênfase na refutação: o cientista falsificacionista testa teorias com o objetivo de encontrar evidências que as refutem.
        \item Impossibilidade de verificação definitiva: teorias nunca podem ser provadas como verdadeiras, apenas corroboradas enquanto não forem refutadas.
    \end{itemize}
    \item Falsificacionismo sofisticado:
    \begin{itemize}
        \item A comparação entre teorias rivais e a busca por teorias mais falsificáveis é fundamental.
        \item As previsões audaciosas que são confirmadas por experimentos conferem um valor significativo a uma teoria.
    \end{itemize}
    \item Limitações do falsificacionismo:
    \begin{itemize}
        \item Dependência da observação: a falibilidade das proposições de observação pode tornar as falsificações inconclusivas.
        \item Complexidade das situações de teste: as teorias científicas muitas vezes envolvem hipóteses auxiliares e condições iniciais complexas.
        \item Inadequação histórica: a aplicação rigorosa do falsificacionismo teria impedido o desenvolvimento de teorias importantes que foram inicialmente refutadas.
    \end{itemize}
\end{itemize}

\subsection{Teorias como Programas de Pesquisa}

\begin{itemize}
    \item Visão holística das teorias:
    \begin{itemize}
        \item As teorias devem ser entendidas como programas de pesquisa estruturados, com uma evolução ao longo do tempo.
        \item A dependência da teoria em relação à observação e a necessidade de orientar pesquisas futuras justificam essa abordagem.
    \end{itemize}
    \item Elementos dos programas de pesquisa:
    \begin{itemize}
        \item Núcleo irredutível: conjunto de hipóteses fundamentais consideradas infalsificáveis.
        \item Cinturão protetor: conjunto de hipóteses auxiliares e condições iniciais ajustadas para proteger o núcleo irredutível.
        \item Heurística negativa: regra metodológica que proíbe a modificação do núcleo irredutível.
        \item Heurística positiva: diretrizes que indicam como o programa deve ser desenvolvido.
    \end{itemize}
    \item Progresso e degeneração dos programas:
    \begin{itemize}
        \item Programas progressivos: levam a descobertas de novos fenômenos, têm previsões corroboradas e mantêm coerência interna.
        \item Programas degenerativos: falham em gerar novas previsões, têm previsões refutadas e recorrem a hipóteses ad hoc.
    \end{itemize}
    \item Comparação e competição entre programas:
    \begin{itemize}
        \item A escolha entre programas rivais deve se basear em sua capacidade de oferecer explicações mais abrangentes, gerar novas previsões bem-sucedidas e manter sua coerência e fertilidade.
    \end{itemize}
\end{itemize}

\subsection{Anarquismo Metodológico de Feyerabend}

\begin{itemize}
    \item Crítica ao método:
    \begin{itemize}
        \item As metodologias tradicionais são incompatíveis com a história da ciência, pois a adesão rígida a regras metodológicas teria impedido o progresso científico.
        \item A única regra que sobrevive ao escrutínio histórico é "vale-tudo", mas isso não significa que qualquer ideia ou prática seja válida na ciência.
    \end{itemize}
    \item Incomensurabilidade:
    \begin{itemize}
        \item Os termos e conceitos de uma teoria adquirem significado dentro do seu próprio arcabouço teórico, o que impede a comparação lógica direta entre teorias rivais.
        \item A escolha final entre teorias dependerá de fatores subjetivos, como os valores e preferências dos cientistas.
    \end{itemize}
    \item Ciência e outras formas de conhecimento:
    \begin{itemize}
        \item A ciência não é necessariamente superior a outras formas de conhecimento, como a arte, a religião ou a magia.
        \item A comparação entre a ciência e outras formas de conhecimento deve ser feita por meio de uma análise cuidadosa de seus objetivos, métodos e resultados.
    \end{itemize}
    \item Relevância do anarquismo metodológico:
    \begin{itemize}
        \item Destaca a importância da criatividade, flexibilidade e abertura a novas ideias no desenvolvimento científico.
        \item Questiona a existência de um método único e universal que garanta o sucesso da ciência.
    \end{itemize}
\end{itemize}

\subsection{Ética e Ciência}

\begin{itemize}
    \item Distinção entre ética e moral:
    \begin{itemize}
        \item A ética, com base na raiz grega ethos, se preocupa com a reflexão crítica sobre os princípios que sustentam as normas e valores morais.
        \item A moral, derivada do latim mos (costumes), se concentra no conjunto de normas, valores, princípios e costumes específicos de uma determinada sociedade ou cultura.
    \end{itemize}
    \item Relação de complementaridade:
    \begin{itemize}
        \item A moral precisa da ética para se repensar e evoluir.
        \item Na ciência, a ética permite avaliar criticamente as normas e práticas da comunidade científica.
    \end{itemize}
    \item Aspectos importantes da ética na ciência:
    \begin{itemize}
        \item Responsabilidade dos cientistas em relação aos impactos sociais e ambientais de suas pesquisas.
        \item Integridade da pesquisa, assegurando honestidade, transparência e rigor.
        \item Conflitos de interesse que podem comprometer a integridade da pesquisa.
        \item Comunicação ética da ciência para o público, promovendo a compreensão pública da ciência.
    \end{itemize}
    \item A ética na ciência é fundamental para garantir que a ciência seja utilizada para o bem da humanidade e do planeta.
\end{itemize}