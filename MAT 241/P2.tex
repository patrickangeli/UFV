\documentclass{article}
\usepackage[margin=1in]{geometry}
\begin{document}

\section*{Resumo de Cálculo III}

\subsection*{1. Funções de várias variáveis}

Uma função de várias variáveis depende de duas ou mais variáveis independentes. Por exemplo:

$$ f(x,y) = x^2 + y^2 $$

\subsection*{2. Limites e continuidade em várias variáveis}

O limite de uma função de várias variáveis é definido de forma similar ao limite de uma função de uma variável:

$$ \lim_{(x,y) \to (0,0)} \frac{xy}{x^2 + y^2} $$

\subsection*{3. Derivadas parciais e direcionais}

Derivadas parciais são calculadas mantendo as outras variáveis constantes:

$$ \frac{\partial f}{\partial x} = \lim_{h \to 0} \frac{f(x+h,y) - f(x,y)}{h} $$
$$ \frac{\partial f}{\partial y} = \lim_{h \to 0} \frac{f(x,y+h) - f(x,y)}{h} $$

A derivada direcional é a taxa de variação da função na direção de um vetor unitário u:

$$ D_u f(x,y) = \nabla f(x,y) \cdot u $$

\subsubsection*{3.1 Regra da Cadeia para Derivadas Parciais}

A regra da cadeia para derivadas parciais é usada quando temos uma função composta. Para uma função z = f(x,y), onde x = g(t) e y = h(t), a regra da cadeia é:

$$ \frac{dz}{dt} = \frac{\partial f}{\partial x} \frac{dx}{dt} + \frac{\partial f}{\partial y} \frac{dy}{dt} $$

Para uma função z = f(x,y), onde x = g(s,t) e y = h(s,t), temos:

$$ \frac{\partial z}{\partial s} = \frac{\partial f}{\partial x} \frac{\partial x}{\partial s} + \frac{\partial f}{\partial y} \frac{\partial y}{\partial s} $$
$$ \frac{\partial z}{\partial t} = \frac{\partial f}{\partial x} \frac{\partial x}{\partial t} + \frac{\partial f}{\partial y} \frac{\partial y}{\partial t} $$

Esta regra é fundamental para resolver problemas complexos envolvendo funções compostas e é amplamente utilizada em aplicações práticas de cálculo multivariável.


\subsection*{4. Gradiente e plano tangente}

O gradiente de uma função é um vetor que contém todas as derivadas parciais da função:

$$ \nabla f(x,y) = \left(\frac{\partial f}{\partial x}, \frac{\partial f}{\partial y}\right) $$

A equação do plano tangente a uma superfície z = f(x,y) no ponto (a,b,f(a,b)) é:

$$ z - f(a,b) = \frac{\partial f}{\partial x}(a,b)(x-a) + \frac{\partial f}{\partial y}(a,b)(y-b) $$

\subsection*{5. Otimização em várias variáveis}

Para encontrar pontos críticos, igualamos todas as derivadas parciais a zero:

$$ \frac{\partial f}{\partial x} = 0, \frac{\partial f}{\partial y} = 0 $$

O teste da segunda derivada para classificação dos pontos críticos:

$$ D = \frac{\partial^2 f}{\partial x^2} \frac{\partial^2 f}{\partial y^2} - \left(\frac{\partial^2 f}{\partial x \partial y}\right)^2 $$

\section{Limites de funções de várias variáveis}

\subsection{Exemplo 1}
\textbf{Problema:} Calcule $\lim_{(x,y) \to (0,0)} \frac{xy}{x^2 + y^2}$

\textbf{Solução:}
\begin{enumerate}
\item Analisamos o comportamento da função ao longo de diferentes caminhos.
\item Caminho 1 (y = x):
$\lim_{x \to 0} \frac{x^2}{x^2 + x^2} = \lim_{x \to 0} \frac{x^2}{2x^2} = \frac{1}{2}$
\item Caminho 2 (y = 0):
$\lim_{x \to 0} \frac{x \cdot 0}{x^2 + 0^2} = 0$
\item Como obtivemos resultados diferentes, concluímos que o limite não existe.
\end{enumerate}

\subsection{Exemplo 2}
\textbf{Problema:} Calcule $\lim_{(x,y) \to (2,2)} (6 - \frac{1}{3}x^2 - \frac{1}{3}y^2)$

\textbf{Solução:}
\begin{enumerate}
\item Substituímos diretamente os valores de x e y:
$\lim_{(x,y) \to (2,2)} (6 - \frac{1}{3}x^2 - \frac{1}{3}y^2) = 6 - \frac{1}{3}(2)^2 - \frac{1}{3}(2)^2 = 6 - \frac{4}{3} - \frac{4}{3} = \frac{10}{3}$
\item O limite existe e é igual a $\frac{10}{3}$.
\end{enumerate}

\section{Derivadas parciais}

\subsection{Exemplo 3}
\textbf{Problema:} Calcule as derivadas parciais de $f(x,y) = x^2y + \sin(xy)$

\textbf{Solução:}
\begin{enumerate}
\item Calculamos $\frac{\partial f}{\partial x}$, tratando y como constante:
$\frac{\partial f}{\partial x} = 2xy + y\cos(xy)$
\item Calculamos $\frac{\partial f}{\partial y}$, tratando x como constante:
$\frac{\partial f}{\partial y} = x^2 + x\cos(xy)$
\end{enumerate}

\subsection{Exemplo 4}
\textbf{Problema:} Calcule as derivadas parciais de $f(x,y) = x^3 + y^2e^x$

\textbf{Solução:}
\begin{enumerate}
\item Calculamos $\frac{\partial f}{\partial x}$:
$\frac{\partial f}{\partial x} = 3x^2 + y^2e^x$
\item Calculamos $\frac{\partial f}{\partial y}$:
$\frac{\partial f}{\partial y} = 2ye^x$
\end{enumerate}

\section{Gradiente e plano tangente}

\subsection{Exemplo 5}
\textbf{Problema:} Encontre o gradiente de $f(x,y,z) = x^2y + yz^2 - 3xz$

\textbf{Solução:}
\begin{enumerate}
\item Calculamos as derivadas parciais:
$\frac{\partial f}{\partial x} = 2xy - 3z$
$\frac{\partial f}{\partial y} = x^2 + z^2$
$\frac{\partial f}{\partial z} = 2yz - 3x$
\item Formamos o gradiente:
$\nabla f = (2xy - 3z, x^2 + z^2, 2yz - 3x)$
\end{enumerate}

\subsection{Exemplo 6}
\textbf{Problema:} Encontre a equação do plano tangente à superfície $z = x^2 + y^2$ no ponto (1, 2, 5)

\textbf{Solução:}
\begin{enumerate}
\item Calculamos as derivadas parciais:
$\frac{\partial z}{\partial x} = 2x$
$\frac{\partial z}{\partial y} = 2y$
\item Avaliamos as derivadas no ponto (1, 2, 5):
$\frac{\partial z}{\partial x}(1,2) = 2$
$\frac{\partial z}{\partial y}(1,2) = 4$
\item A equação do plano tangente é:
$z - 5 = 2(x - 1) + 4(y - 2)$
\end{enumerate}

\section{Otimização}

\subsection{Exemplo 7}
\textbf{Problema:} Encontre os pontos críticos de $f(x,y) = x^2 + y^2 - 2x - 4y$

\textbf{Solução:}
\begin{enumerate}
\item Calculamos as derivadas parciais:
$\frac{\partial f}{\partial x} = 2x - 2$
$\frac{\partial f}{\partial y} = 2y - 4$
\item Igualamos as derivadas parciais a zero:
$2x - 2 = 0 \implies x = 1$
$2y - 4 = 0 \implies y = 2$
\item O ponto crítico é (1, 2).
\end{enumerate}

\subsection{Exemplo 8}
\textbf{Problema:} Encontre os extremos locais de $f(x,y) = x^2 + xy + y^2 - 3x - 3y$

\textbf{Solução:}
\begin{enumerate}
\item Calculamos as derivadas parciais:
$\frac{\partial f}{\partial x} = 2x + y - 3$
$\frac{\partial f}{\partial y} = x + 2y - 3$
\item Igualamos as derivadas parciais a zero:
$2x + y - 3 = 0$
$x + 2y - 3 = 0$
\item Resolvemos o sistema:
$x = y = 1$
\item Calculamos a matriz Hessiana:
$H = \begin{bmatrix} 2 & 1 \\ 1 & 2 \end{bmatrix}$
\item Como $\det(H) = 3 > 0$ e $\frac{\partial^2 f}{\partial x^2} = 2 > 0$, o ponto (1, 1) é um mínimo local.
\end{enumerate}

\section{Multiplicadores de Lagrange}

\subsection{Exemplo 9}
\textbf{Problema:} Maximize $f(x,y) = xy$ sujeito à restrição $x^2 + y^2 = 1$

\textbf{Solução:}
\begin{enumerate}
\item Formamos a função Lagrangiana:
$L(x,y,\lambda) = xy - \lambda(x^2 + y^2 - 1)$
\item Calculamos as derivadas parciais:
$\frac{\partial L}{\partial x} = y - 2\lambda x = 0$
$\frac{\partial L}{\partial y} = x - 2\lambda y = 0$
$\frac{\partial L}{\partial \lambda} = x^2 + y^2 - 1 = 0$
\item Resolvemos o sistema:
$x = y = \frac{1}{\sqrt{2}}$ ou $x = y = -\frac{1}{\sqrt{2}}$
\item O máximo ocorre em $(\frac{1}{\sqrt{2}}, \frac{1}{\sqrt{2}})$ e o mínimo em $(-\frac{1}{\sqrt{2}}, -\frac{1}{\sqrt{2}})$.
\end{enumerate}

\subsection{Exemplo 10}
\textbf{Problema:} Minimize $f(x,y) = x^2 + y^2$ sujeito à restrição $x + y = 1$

\textbf{Solução:}
\begin{enumerate}
\item Formamos a função Lagrangiana:
$L(x,y,\lambda) = x^2 + y^2 - \lambda(x + y - 1)$
\item Calculamos as derivadas parciais:
$\frac{\partial L}{\partial x} = 2x - \lambda = 0$
$\frac{\partial L}{\partial y} = 2y - \lambda = 0$
$\frac{\partial L}{\partial \lambda} = x + y - 1 = 0$
\item Resolvemos o sistema:
$x = y = \frac{1}{2}$
\item O ponto $(\frac{1}{2}, \frac{1}{2})$ é o mínimo da função sujeito à restrição.
\end{enumerate}



\end{document}