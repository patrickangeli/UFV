\date{08-10-2024}

Exemplo: \mathexpr{\overrightarrow{u} = <4,0,3> e \overrightarrow{v} = <-2,1,5>}.
Determine: \\
\begin{itemize}
    \item[a.] \mathexpr{}
    \item[b. ]  
\end{itemize}

\subsubsection{Prorpiendades}: 

Sejam $\overrightarrow{u}, \overrightarrow{v}$ e $\overrightarrow{w}$ vetores do $\mathbb{R}$ e $a,b \in \mathbb{R}$.
Então: \\
\begin{itemize}
    \item[a.] $\overrightarrow{u} + \overrightarrow{v} = \overrightarrow{v} + \overrightarrow{u}$
    \item[b.] $(\overrightarrow{u}+ \overrightarrow{v}) + \overrightarrow{w} = \overrightarrow{u} + (\overrightarrow{v} + \overrightarrow{w})$  
    \item[c.] $\overrightarrow{u} + \overrightarrow{0} = \overrightarrow{u}$ 
    \item[d.] $\overrightarrow{u} + (-\overrightarrow{u}) = \overrightarrow{0}$ 
    \item[e.] $a (\overrightarrow{u} + \overrightarrow{v}) = a \overrightarrow{u} + a \overrightarrow{v}$
    \item[f.] $(a+b) \overrightarrow{u} = a \overrightarrow{u} + b \overrightarrow{u}$ 
    \item[g.] $(ab) \overrightarrow{u} = a (b \overrightarrow{u})$
    \item[h.] $1 \overrightarrow{u} = \overrightarrow{u}$ 
\end{itemize}

\subsubsection{Propriedades (Normas):}
\begin{itemize}
    \item[a.] $\norma{\overrightarrow{u}} \geq$ e $\norma{\overrightarrow{u}} = 0 \leftrightarrow \norma{u} = \norma{0}$
    \item[b.] $\norma{k \overrightarrow{u}} = |k| \norma{\overrightarrow{u}}$ 
    \item[c.] $ \|\vec{u} + \vec{v}\| \leq \|\vec{u}\| + \|\vec{v}\| \quad (\text{desigualdade do triângulo}$ 
\end{itemize}

$\text{Obs: Dado } \vec{u} = \vec{0}, \text{posso obter um novo vetor que é nulo.}$

\begin{align*}
    &\text{Temos } \vec{u} = \frac{\vec{u}}{\|\vec{u}\|}, \text{que} \\
    &\|\vec{u}\| = 1 \implies \|\lambda \vec{u}\| = \|\vec{u}\| = K \|\vec{u}\| \\
    &= 1 \implies K = \frac{1}{\|\vec{u}\|}
\end{align*}


Em $\mathbb{R}^3$, demonstremos por:
\begin{align*}
    \vec{i} &= \langle 1, 0, 0 \rangle \\
    \vec{j} &= \langle 0, 1, 0 \rangle \\
    \vec{k} &= \langle 0, 0, 1 \rangle
\end{align*}

Dai, seja $<x, y, z> \in \mathbb{R}^3$ então: TA ERRADO

$\langle \vec{x}, \vec{x} \rangle = x\langle \vec{e}_1, \vec{e}_1 \rangle + y\langle \vec{e}_1, \vec{e}_1 \rangle + z\langle \vec{e}_1, \vec{e}_1 \rangle$

$\vec{e}_i \in \{(1, 0, 0), (0, 1, 0), (0, 0, 1)\}$

$\Rightarrow \vec{x} \cdot \vec{x} = x^2 + y^2 + z^2$

Produto escalar

Def: Sejam $\vec{u} = (u_1, u_2, u_3)$ e $\vec{v} = (v_1, v_2, v_3)$. 

Definimos o produto escalar (interno)

$\vec{u} \cdot \vec{v} = u_1v_1 + u_2v_2 + u_3v_3$.
    
