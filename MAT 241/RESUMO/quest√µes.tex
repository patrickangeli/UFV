\documentclass{article} 

\usepackage{amsmath}
\usepackage[shortlabels]{enumitem}

\begin{document}

\section*{Question 57}
Escreva uma equação do plano que contém o ponto $A = (1, -2, 3)$ e é perpendicular a cada um dos planos $2x + y - z = 2$ e $x - y - z = 3$.

\section*{Question 55}
Determine a distância do ponto $A = (2, 1, 3)$ a cada um dos planos:
\begin{enumerate}[a)]
\item $x - 2y + z = 1$
\item $x + y - z = 0$
\item $x - 5z = 8$
\end{enumerate}

\section*{Question 56}
Determine:
\begin{enumerate}[a)]
\item a distância do ponto $(5, 4, 7)$ à reta $r: \begin{cases} x = 1 + 5t \\ y = 2 - t \\ z = t \end{cases}, t \in \mathbb{R}$
\item a distância do ponto $(1, 2, -1)$ à reta $r: \begin{cases} x = 1 + 2t \\ y = 5 - t \\ z = -2 + 3t \end{cases}, t \in \mathbb{R}$
\item a distância do ponto $(2, 3, 5)$ a cada um dos eixos do sistema de coordenadas.
\end{enumerate}

\section*{Question 4}
Julgue cada item abaixo como verdadeiro ou falso, justificando com um argumento lógico ou com um contraexemplo. Respostas sem justificativas não serão consideradas.

\begin{enumerate}[a)]
\item (6 pontos) O triângulo determinado pelos pontos $A = (0, 0, 0)$, $B = (1, \sqrt{2}, 1)$ e $C = (2, 0, 0)$ é um triângulo equilátero.
\item (6 pontos) O raio da circunferência, obtida pela interseção da esfera de equação $x^2 + y^2 + z^2 = 4$ com o plano $\pi$ de equação $x + y + z = 1$, é $r = \sqrt{\frac{11}{3}}$.
\item (6 pontos) Se $4A^2 + B^2 - 4C^2 < 0$, então a equação $x^2 - Ax + 4y^2 + By - z^2 + Cz = 0$ representa um hiperbolóide de uma folha.
\item (6 pontos) O ponto $D = (0, 4, 1)$ pertence ao plano determinado pelos pontos $A = (1, 0, 2)$, $B = (-2, 0, 1)$ e $C = (-1, 2, 1)$.
\item (6 pontos) Se $\vec{u} \times \vec{v} = \vec{u} \times \vec{w}$, com $\vec{u} \neq \vec{0}$, então $\vec{v} = \vec{w}$.
\end{enumerate}

\section*{Questions 17-22 (Vector problems)}
\begin{enumerate}
\item[17.] Sejam $\vec{u}$ e $\vec{v}$ vetores, com ângulo entre si medindo $\theta = \frac{\pi}{6}$ e tais que $(\vec{u}, \vec{v}) = 2$. Determine a área do triângulo que tem os vetores $\vec{u}$ e $\vec{v}$ como lados adjacentes.

\item[18.] Se $\vec{u}$ e $\vec{v}$ são vetores tais que $\|\vec{u} + \vec{v}\| = 10$ e $\|\vec{u} - \vec{v}\| = 8$, determine $(\vec{u}, \vec{v})$.

\item[19.] Sejam $\vec{u}$ e $\vec{v}$ vetores unitários tais que $(\vec{u}, \vec{v}) = \frac{1}{2}$. Determine $(\vec{u} + \vec{v}, \vec{u} - \vec{v})$ e $\|\vec{u} + \vec{v}\|$.

\item[20.] Seja $\vec{v} = (1, -5, 3)$. Determine o vetor $\vec{w}$, tal que $\|\vec{w}\| = 10$, e que tem a mesma direção e o sentido contrário ao $\vec{v}$.

\item[21.] Obtenha $\vec{v}$ tal que $\vec{v} \times \vec{j} = \vec{k}$ e $\|\vec{v}\| = \sqrt{5}$.

\item[22.] Sejam $\vec{u} = a\vec{i} + 2\vec{j} + \vec{k}$ e $\vec{v} = \vec{i} + \vec{j} - 2\vec{k}$. Sabendo-se que o ângulo entre $\vec{u}$ e $\vec{v}$ é obtuso, determine o valor de $a$ de modo que a área do paralelogramo determinado pelos vetores $\vec{u}$ e $\vec{v}$ seja $\sqrt{90}$.
\end{enumerate}

\section*{Question 3 (Vector orthogonality)}
Seja $\vec{u}$ um vetor ortogonal a $\vec{v}$ e $\vec{w}$. Sabendo-se que $\vec{v}$ e $\vec{w}$ formam um ângulo de $30°$ e que $\|\vec{u}\| = 6$, $\|\vec{v}\| = 3$ e $\|\vec{w}\| = 3$, calcule $(\vec{u}, \vec{v} \times \vec{w})$.

\section*{Question 68 (Plane equations)}
Determinar a equação geral dos planos nos seguintes casos:
\begin{enumerate}[a)]
\item passa pelo ponto $D = (1, -1, 2)$ e é ortogonal ao vetor $\vec{v} = (2, -3, 1)$
\item possui o ponto $A = (1, 2, 1)$ e é paralelo aos vetores $\vec{u} = \vec{i} + \vec{j} - \vec{k}$ e $\vec{v} = \vec{i} + \vec{j} - 2\vec{k}$
\item passa pelos pontos $A = (2, 1, 5)$, $B = (3, 1, 3)$ e $C = (4, 2, 3)$
\item passa pelo ponto $E = (1, 2, 2)$ e contém os vetores $\vec{u} = (2, -1, 1)$ e $\vec{v} = (-3, 1, 2)$
\item possui o ponto $P = (2, 1, 3)$ e é paralelo ao plano $\pi$
\item contém as retas $r: \frac{x-7}{3} = \frac{y-2}{2} = \frac{1-z}{2}$ e $s: \frac{x-1}{3} = \frac{y+2}{3} = \frac{z-5}{4}$
\item contém as retas $r: \frac{x}{2} - y + 1 = z + 3$ e $s: \frac{x+1}{4} = \frac{y-2}{2} = \frac{z}{2}$
\item contém as retas $r: \begin{cases} x = -3+t \\ y = -t \\ z = 4 \end{cases}, t \in \mathbb{R}$ e $s: \frac{x+2}{2} = \frac{2-y}{2} = z = 0$
\item contém a reta $r: \frac{x-1}{2} = \frac{y}{2} = z - 1$ e é paralelo à reta $s: \frac{x-3}{2} = 2-y = \frac{z-4}{4}$
\end{enumerate}

\section*{Question 72 (Spheres)}
Encontre o centro e o raio das esferas:
\begin{enumerate}[a)]
\item $x^2 + y^2 + z^2 + 4x - 4z = 0$
\item $2x^2 + 2y^2 + 2z^2 + x + y + z = 9$
\end{enumerate}

\section*{Question 82 (Surface equation)}
Determine a equação da superfície definida pelo conjunto dos pontos $P = (x, y, z)$ tais que a distância de $P$ ao eixo dos $y$ é $\frac{3}{4}$ da distância de $P$ ao plano $xz$. Identifique a superfície.

\section*{Question 2 (Planes)}
Considere os planos:
\[\pi_1: x - y - 2z = 3 \quad \text{e} \quad \pi_2: -2x - y + z = 5\]

\begin{enumerate}[a)]
\item (5 pontos) Caso exista, determine as equações paramétricas da reta de interseção dos planos $\pi_1$ e $\pi_2$
\item (5 pontos) Determine o ângulo formado por $\pi_1$ e $\pi_2$
\item (10 pontos) Seja $s$ a reta de equações paramétricas dadas por
\[s: \begin{cases} x = 2 + 2t \\ y = -3 - 4t \\ z = -2 - 3t \end{cases}, t \in \mathbb{R}\]
Determinar, caso existam, os pontos do espaço que estão localizados sobre a reta $s$ e que distam $\sqrt{6}$ unidades do plano $\pi_1$.
\end{enumerate}

\end{document}
